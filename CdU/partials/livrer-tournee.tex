\begin{UseCase}{Livrer une tournée}

\UseCaseActor{Ronde, Drone}

\UseCaseContext{Le système recalcule la(les) tournée(s) une fois qu'une
    commande arrive et la(les) livrer à l'aide des drones.}

\begin{UseCasePre}
    \begin{enumerate}
        \item Une demande de livraison ou une annulation de demande de livraison a été faite.
    \end{enumerate}
\end{UseCasePre}

\begin{UseCasePost}
    \begin{enumerate}
        \item Les commandes livrées et un retour de l'état du drone (bon état,
            mauvais état, drone perdu ...).
    \end{enumerate}
\end{UseCasePost}

\begin{UseCaseScenario}
    \begin{enumerate}
        \item Le système crée ou recalcule les tournées (tient compte des
            commandes et des drones disponible).
        \item Le système calcule un facteur de rentabilité pour chaque tournée.
            Ce coefficient est calculé en faisant le rapport de la quantité de
            produit transport sur la capacité du drone.
        \item Lorsqu'un facteur de rentabilité est supérieur à un chiffre
            donné (\textbf{80\%} par exemple), la tournée correspondante est livrée :
            \begin{enumerate}[label={\theenumi\alph*.}]
                \item Les commandes sont préparées par un agent (humain ou
                    robotique).
                \item Elles sont placées dans le drone, par ordre d'arrivé chez
                    le client.
                \item Le système envoie le drone.
                \item Le système attend le retour du drone.
                \item Le système supprime la tournée correspondant de la liste
                    de tournées à livrer.
                \item Revenir à l'étape 1.
            \end{enumerate}
        \item Lorsqu'un facteur de rentabilité est inférieur à ce chiffre
            donné, on place une alarme sur la tournée. Cette alarme prévient le
            système lorsque une commande de la tournée ne sera pas livrée à
            temps (délai d'attente fixé, par exemple à \textbf{6h}) si on
            attend davantage que le coefficient s'améliore.
    \end{enumerate}
\end{UseCaseScenario}

\begin{UseCaseExtension}
    \begin{enumerate}
        \item[2.a] Il n'y a plus assez de drone. La capacité de chaque drone
            est dépassée :
            \begin{enumerate}
                \item On crée une nouvelle tournée.
                \item On transfère les commandes de la tournée dépassant la
                    limite à cette nouvelle tournée. \textit{Répéter tant que
                        la tournée citée dépasse la capacité du drone.}
                \item On notifie le client si une de ses commandes ne sera pas
                    livrée à temps.
                \item On livre les commandes restante en passant à l'étape 3.
            \end{enumerate}
        \item[3.a.a] Certains produits ne sont plus disponible en stock pour
            réaliser une commande :
            \begin{enumerate}
                \item On modifie la commande du client en enlevant tous les
                    articles non disponible.
                \item On notifie le client que sa commande a dû être modifiée.
                \item On actualise les stocks.
                \item On reprend l'exécution du scénario principal en ignorant
                    les articles non disponible (passer à l'étape 3b).
            \end{enumerate}
        \item[3.d.a] Le drone retourne en mauvais état.
            \begin{enumerate}
                \item Envoyer ce drone au service technique.
                \item Créer un fichier log sui sera lu plus tard par un expert.
            \end{enumerate}
        \item[3.d.b] Le drone n'est pas revenu, alors qu'il est attendu depuis
            2h (c'est à dire un temps de livraison beaucoup trop long) :
            \begin{enumerate}
                \item Envoyer une équipe à sa recherche en suivant son trajet.
                \item Si le drone est trouvé, le placer au service technique
                \item Si le drone n'est pas trouvé, prévenir la police.
                \item Créer un fichier log qui sera lu plus tard par un expert.
            \end{enumerate}
        \item[4.a] Une alarme arrive à expiration :
            \begin{enumerate}
                \item On exécute l'étape 3 même si le coefficient de
                    rentabilité est inférieur au chiffre donné.
            \end{enumerate}
    \end{enumerate}
\end{UseCaseExtension}

\end{UseCase}
