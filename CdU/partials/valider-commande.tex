\begin{UseCase}{Valider et lancer une commande}

\UseCaseActor{Utilisateur}

\UseCaseContext{L'utilisateur reçoit la commande. Il vérifie la liste des
    produits de la commande. Il signale les éventuelles anomalie de commande.
    S'il considère que tout est en règle, il valide et la lance la commande.}

\begin{UseCasePre}
    \begin{enumerate}
        \item Le système est bien initialisé avec les seuils des produits fixés
            par l'utilisateur (voir cas d'utilisation
            \ref{sec:modifier-param-balance}).
        \item La liste des produits à être commandés est établie (voir cas
            d'utilisation \ref{sec:assigner-produit-balance}).
    \end{enumerate}
\end{UseCasePre}

\begin{UseCasePost}
    \begin{enumerate}
        \item Les produits affectés à la liste de commande sont prêt à être
            traités par l'entrepôt.
    \end{enumerate}
\end{UseCasePost}

\begin{UseCaseScenario}
    \begin{enumerate}
        \item L'utilisateur reçoit la liste des produits à être commandés.
        \item L'utilisateur vérifie la liste des produits.
        \item L'utilisateur modifie manuellement la commande [optionnel].
        \item Il signale d'éventuelles anomalies liées à la commande [optionnel].
        \item Il valide et lance la commande.
    \end{enumerate}
\end{UseCaseScenario}

\begin{UseCaseExtension}
    \begin{enumerate}
        \item[1.a] Aucune commande n'a été notifiée :
        \begin{enumerate}
            \item Attendre un délai de \textbf{10 à 20 minutes}.
            \item Passé ce délai, l'utilisateur signale le problème à l'entreprise.
            \item Attendre la nouvelle notification du contrôleur et continuer à l'étape 2.
        \end{enumerate}
        \item[3.a] Ajout multiple du même produit :
        \begin{enumerate}
            \item L'utilisateur décoche les produits en trop.
            \item Le CdU reprend à l'étape 4.
        \end{enumerate}
    \end{enumerate}
\end{UseCaseExtension}

\end{UseCase}
