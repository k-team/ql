\begin{UseCase}{Mise à jour d'une commande}

\UseCaseActor{Système client}

\UseCaseContext{Quand le système client reçoit une notification de la part
    d'une balance, signalant le manque du produit qui lui est associé, il
    vérifie la disponibilité en stock du produit. Il affecte ensuite ce produit
    à la liste de commandes courante du client. Il signale la commande à
    l'utilisateur pour être validée via une notification.}

\begin{UseCasePre}
    \begin{enumerate}
        \item Le système est bien initialisé avec les balances configurées par
            l'utilisateur (voir cas d'utilisation :
            \ref{sec:assigner-produit-balance}).
    \end{enumerate}
\end{UseCasePre}

\begin{UseCasePost}
    \begin{enumerate}
        \item Les produits affectés à la liste de commande sont en attente de
            validation par l'utilisateur.
    \end{enumerate}
\end{UseCasePost}

\begin{UseCaseScenario}
    \begin{enumerate}
        \item L'utilisateur consomme des produits et les quantités descendent
            en dessous des seuils associés.
        \item Les balances associées préviennent le système client de la manque
            en produits. Cette étape peut être répétee plusieurs fois.
        \item Le système client vérifie les produits signalés avec la
            disponibilité des produits en stock. Le système client peut répéter
            cette étape plusieurs fois.
        \item Le système client ajoute les produits à la liste de commande de
            l'utilisateur. Le système client peut répéter cette étape plusieurs
            fois.
        \item Le système client notifie l'utilisateur de la mise à jour de la
            commande.
    \end{enumerate}
\end{UseCaseScenario}

\begin{UseCaseExtension}
    \begin{enumerate}
        \item[1.a] Aucun seuil n'est fixé :
            \begin{enumerate}
                \item Le système client notifie l'utilisateur du problème.
                \item L'utilisateur fixe le seuil des balances associées aux
                    produits : le scénario se termine par un échec \textit{OU}
                    reprendre le scénario à l'étape 1.
            \end{enumerate}
        \item[1.b] Les balances sont défectueuses :
            \begin{enumerate}
                \item Le système client notifie l'utilisateur du problème.
                \item L'utilisateur prend contact avec l'entreprise.
                \item Les balances sont réparées et réinitialisées : le
                    scénario se termine par un échec \textit{OU} reprendre le
                    scénario à l'étape 1.
            \end{enumerate}
        \item[2.a] Aucun signal n'est envoyé au système client :
            \begin{enumerate}
                \item L'utilisateur prend contact avec l'entreprise.
                \item Les balances sont réparées et réinitialisées : le
                    scénario se termine par un échec \textit{OU} reprendre le
                    scénario à l'étape 1.
            \end{enumerate}
        \item[3.a] Le produit n'est pas disponible :
            \begin{enumerate}
                \item Le système client notifie l'utilisateur que les articles
                    ne sont pas en stock.
                \item Le scénario continue à l'étape 4.
            \end{enumerate}
        \item[4.a] Ajout multiple du même produit (envoi continu d'un signal) :
            \begin{enumerate}
                \item Le système client notifie l'utilisateur de l'anomalie si
                    le nombre de signaux émis par la balance est supérieur à
                    \textit{10}.
                \item L'utilisateur modifie manuellement la commande.
                \item La commande est mise à jour.
                \item Le scénario continue à l'étape 5.
            \end{enumerate}
        \item[5.a] Aucune notification de commande n'a été reçue :
            \begin{enumerate}
                \item Attendre un délai de 10 à 20 minutes.
                \item Passé ce délai, l'utilisateur prend contact avec
                    l'entreprise et l'informe du problème.
                \item Attendre une nouvelle notification du système client.
            \end{enumerate}
    \end{enumerate}
\end{UseCaseExtension}

\end{UseCase}
