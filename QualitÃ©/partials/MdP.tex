\begin{UseCase}{Modèle du Protocole}

\UseCasDescription{
Le MdP découle du MdC et MdE. Les 3 modèles se focalisent sur le même système. Le MdP est le un diagramme d’état - transistion. On décrit les états du sytèmes et les transistions pour passer d’un état à un autre.
} 

\UseCaseModele{
Chaque changement d’état est induit par un message. Ce qu’il veut dire que l’on passe d’un état à un autre que si cela est déterminer par une action. Le non determinisme est à proscrire ou à limiter fortement dans ce modèle.\\

Chaque sous-système doit être décrit dans un modèle de protocole, et tous ses états doivent y être décrits. On a donc un MdP pour chaque MdC( et a fortiori pour chaque MdE). Chaque message provient d’une interaction et le système utilisé pour le MdE.\\

Un transition qui sur son état d’origine signifie pas que le système n’a pas evoluer. Dans l’exemple, l’action “MonnaieInsérée”  permet d'incrémenter la variable “argentIntroduit” (par exemple). \\

Le système ne doit arriver dans un état sans qu’il puisse changer d’état. Il doit avoir “le choix” de passer d’un état à l’autre selon les actions qui l’affecte.\\
}

\UseCaseEntreModele{
Chaque sous-système doit être décrit dans un modèle de protocole, et tous ses états doivent y être décrits.\\
Chaque message provient d’une interaction et le système utilisé pour le MdE.\\
Les sous-systemes d’un MdP sont les même que ceux des MdC ou MdE
}

\end{UseCase}