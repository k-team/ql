\documentclass[10pt, a4paper]{article}

\usepackage[frenchb]{babel}
\usepackage[T1]{fontenc}
\usepackage{textcomp}
\usepackage[utf8x]{inputenc}
\usepackage{graphicx}
\usepackage{float}
\usepackage{hyperref}
\usepackage{enumitem}

% Format du document
\renewcommand{\familydefault}{\sfdefault}
\usepackage[top = 1.25cm, bottom = 1.25cm, left=2.5cm, right=2.5cm]{geometry}

\newcommand{\HRule}{\rule{\linewidth}{0.5mm}}
\newcommand{\Document}[1]{
    \begin{titlepage}
        % Titre
        \vspace*{\fill}
        \centering
        \HRule \\[0.4cm]
        {\huge \bfseries Projet Qualité Logicielle : \\ #1}
        \HRule \\[1.5cm]

        % Auteurs
        \begin{minipage}{0.8\textwidth}
            \centering
            \large
            Équipe H4314 -- \today
        \end{minipage}

        \vfill

        \flushleft
        \small
        {\bfseries Chef de projet} \\
        Jean-Marie COMETS \\[0.5cm]

        {\bfseries Membres de l'équipe} \\
        Franck MPEMBA BONI: responsable qualité \\
        Pierre TURPIN \\
        Samuel CARENSAC \\
        Grégoire CATTAN \\
        Van PHAN HAU \\
        Iler VIRARAGAVANE
    \end{titlepage}

    \tableofcontents
    \newpage
}


\newenvironment{Phase}[3]{
    \subsection{#1}
    \noindent
    \textbf{Fournitures :} #2 \\
    \textbf{Livrables :} #3 \\
}{
}

\newenvironment{SubPhases}{
    \textbf{Sous-phases :}
    \begin{enumerate}
}{
    \end{enumerate}
}

\begin{document}

\Document{Dossier d'Initialisation}

\section{Contexte et objectifs de l'étude}

L'entreprise Ronde est chargée d'installer un système centralisé de livraison à
la personne dans la ville de Lyon pour améliorer \textit{l'intelligence
    urbaine}, se plaçant dans un contexte de connectivité moderne. L'étude doit
permettre d'automatiser certaines tâches humaines pour libérer du temps pour
les clients concernés et limiter le nombre de trajets chez des magasins de
distribution, en les remplaçant par des livraisons directes.

\section{Description de l'équipe}

L'équipe "K", chargée de l'étude est composée des membres suivants :

\begin{itemize}
\item Jean-Marie Comets, chef d'équipe
\item Franck Mpemba Boni, responsable qualité
\item Samuel Carensac
\item Pierre Turpin
\item Iler Viraragavane
\item Grégoire Cattan
\item Phan Van Hau
\end{itemize}

\section{Description des livrables}

\begin{itemize}
\item Dossier d'initialisation : ce document.
\item Architecture du système : modélisation du système envisagé accompagné d'une description textuelle de chacun des éléments non-trivaux.
\item Cas d'utilisation : ensemble de documents donnant une description textuelle de chacunes des utilisations du système, décrivant notamment les cas idéals ainsi que les cas d'erreurs, leur interprétation et leur gestion.
\item Modèles de l'environnement : ensemble de schémas accompagnés d'une description textuelle, détaillant les différents sous-systèmes et leurs interfaces externes.
\item Modèles des concepts
\item Modèles de protocole
\item Spécifications OCL des opérations
\end{itemize}

\section{Découpage en phases}

\begin{enumerate}
\item Identification des besoins - Modèles des cas d'utilisation
\item Conception de la solution envisagée - Architecture du système
\item Définition des opérations - Modèles de l'environnement
\item Définition des entités - Modèles des concepts
\item Description des protocoles - Modèles de protocole
\item Description des opérations - Spécifications OCL des opérations
\end{enumerate}

\begin{Phase}{Identification des besoins}{cahier des charges client}{modèles des cas d'utilisation}
\begin{SubPhases}
\item Initialisation
\item Définition du système
* livrable intermédiaire : description textuelle des besoins
\item Identification des cas d'utilisation
* livrables intermédiaires : listing des cas d'utilisation et des acteurs du système
\item Rédaction des cas d'utilisation
\item Bilan
\end{SubPhases}
\end{Phase}

\begin{Phase}{Conception de la solution envisagée}{modèles des cas d'utilisation et description textuelle des besoins}{architecture du système}
\begin{SubPhases}
\item Initialisation
\item Identification des acteurs système
\item Spécification des interfaces de communication entre sous-systèmes
* livrables intermédiaires : diagrammes détaillés des interfaces de communication
\item Spécification du matériel nécessaire
* livrables intermédiaires : description textuelle du matériel envisagé
\item Bilan
\end{SubPhases}
\end{Phase}

\begin{Phase}{Définition des opérations}{modèles des cas d'utilisation et architecture du système}{modèles de l'environnement}
%\begin{SubPhases}
%\item Initialisation
%\item Bilan
%\end{SubPhases}
\end{Phase}

\begin{Phase}{Définition des entités}{listing des acteurs et architecture du système}{modèles des concepts}
%\begin{SubPhases}
%\item Initialisation
%\item Bilan
%\end{SubPhases}
\end{Phase}

\begin{Phase}{Description des opérations}{définition des opérations, modèles des cas d'utilisation et architecture du système}{Spécifications OCL des opérations}
%\begin{SubPhases}
%\item Initialisation
%\item Bilan
%\end{SubPhases}
\end{Phase}

\section{Estimation de la charge par phase}

\begin{center}
    \begin{tabular}{lr}
        \textbf{Phase} & \textbf{Charge} \\
    Identification des besoins & 20 \% \\
    Conception de la solution envisagée & 10 \% \\
    Définition des opérations & 15 \% \\
    Définition des entités & 15 \% \\
    Modèles de protocole & 20 \% \\
    Description des opérations & 20 \%
    \end{tabular}
\end{center}

\section{Analyse des risques}

Nous concevons à la fois le système d'approvisionnement intelligent et
l'application centralisant les commandes client. Nous ne sommes pas
responsables de l'utilisation du réseau par les drones, et donc les risques
liés à la navigation des drones ne nous concernent pas. \\

Par contre, nous sommes responsables des livraisons aux clients et de l'état
final de leurs commandes, donc un premier risque identifié est le
disfonctionnement du chargement des drones et le déchargement du colis dans le
réceptacle du client. Dans le même esprit, nous devons aussi maintenir les
drones dans un état correct lors du chargement et de la livraison. \\

Ensuite, nous devons assurer le bon fonctionnement du système client dans le
cas d'un disfonctionnement matériel ou logiciel. Ceci implique que nous devons
gérer la signalisation des problèmes observés par le client (hotline,
signalisation transmise au système centralisé, etc...). \\

Enfin, nous devons garantir le controle des drones dans le cas d'une panne du
système centralisé à l'entrepôt, qui pourrait impliquer un dommage des drones.
Les clients et les systèmes clients doivent notamment être informés de la panne
à ce moment.

\section{Description des moyens mis en œuvre pour la gestion de projet}

Pour la gestion des tâches, nous avons décidé d'utiliser une application web
fonctionnant sur un système de tickets : \href{https://trello.com}{Trello}. Celle-ci
intègre des notifications par email et des commentaires sur tickets, ainsi que
la possibilité d'attacher des fichiers aux tickets, permettant le suivi en
continu de l'évolution d'une tâche (une tâche correspond à un ou plusieurs
tickets). \\

En ce qui concerne la rédaction, nous avons utilisé Google Drive avec des
extensions pour la modélisation (LucidChart notamment). Grâce au partage
immédiat avec l'équipe et à l'édition en temps réel, nous pouvons travailler le
plus efficacement possible sur les différentes tâches de rédaction, en
observant l'évolution au fur et à mesure du document. La fonctionnalité d'envoi
groupé à tous les collaborateurs est d'ailleurs très utile pour une
communication rapide envers les membres d'équipe concernés par un ou plusieurs
documents. \\

Enfin pour les réunions d'équipe, nous avons utilisé la méthode classique du
tableau noir pour faire un brainstorming des idées et clarifier la
compréhension du système de chacun.

\end{document}
