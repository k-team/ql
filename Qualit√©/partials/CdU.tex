\begin{UseCase}{Cas d'utilisation}

\UseCaseDescription{
		Les cas d’utilisation sont la description d’une suite d’interactions entre des acteurs (humains ou non) et un système. Ils sont représentés graphiquement par des diagrammes de cas d’utilisation. Les cas d’utilisation doivent repondre à un besoin clairement expliciter dans le cahier des charges.

}


\begin{figure}
   \centering
   \includegraphics[width=\textwidth]{../imagesCdU.png}
   \caption{Schéma du modèle de l'environnement}
\end{figure}

\UseCaseModele{
Chaque cas d'utilisation doit être structuré de façon à constituer une spécification complète et intuitive des fonctionnalités.\\

Il faut veiller à ce que les extensions/dysfonctionnement découlent toujours d’une étape du scénario établi.\\

Il faut essayer de trouver un bon niveau de détails pour le CdU. On doit être ni trop précis, ni trop vague. La difficulté est de pouvoir être concis sans rentrer dans trop de details. Il est conseillé de commencer de manière abstraite et d’affinité ces modèles au fur et à mesure. Si on arrive à plus de 15 CdU, on estime que la précisions est trop importante. Cela signifie que certaine tache peuvent sûrement être regroupé entre elle.\\

Le CdU doit être compréhensible par le client donc le choix des nom des acteurs et de l’ensemble des éléments du schéma doit être clair et simple. Le langage ne doit pas être technique.\\

Il ne faut pas décrire les interaction entre les acteurs. Un CdU décrit les interactions entre les acteurs et le système. \\

Attention à bien choisir les acteurs. Un acteur ne doit pas être interne au système. Il doit profiter de l’utilisation du système, et interagir directement avec lui. Il est plus un rôle,
un profil, qu’une entité physique concrète.

}

\UseCaseEntreModele{
Si un cas d’utilisation possède une pré-condition, il faut s’assurer qu’il existe un autre cas d’utilisation  qui possède une post -condition identique.\\

Les cas d'utilisation doivent rester autant que possible indépendants les uns des autres. Il faut éviter de se perdre dans les détails quant aux interférences, à la concurrence et aux conflits entre cas d'utilisation.

}

\end{UseCase}

