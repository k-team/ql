\begin{LQ}{Modèle des Opérations}

\LQDescription{
    Le modèle des opération modélise l'ensemble des conditions qui causent et
    qui sont causées par la communication entre acteurs et système. Il
    représente ainsi pour chaque message l'enchaînement d'événement induit par
    les pré et post conditions de son envoie.Les post conditions correspond à
    l'algorithme de fonctionnement de notre système dépendant des divers
    messages pouvant intervenir lors de notre opération.
}


\textbf{Description du modèle textuelle}\\
Cette étude doit suivre un schéma précis constituée comme suit:
 \\
\begin{description}
\item[Opération :] Système::Opération
\item[Acteur du MdE :] Nom de l’acteur du MdE rattaché à cette opération
\item[CdU :] Nom du CdU rattaché à cette opération
\item[Message :] Acteur{messages envoyé par cette acteur}
\item[Pré :] Algorithme définissant les pré-conditions
\item[Post :] Algorithme définissant les post
\end{description}

\textbf{Exemple du modèle textuelle}\\
Cette étude doit suivre un schéma précis constituée comme suit:
 \\
\begin{description}
\item[Opération :]  DAB::éjecterMonnaie() 
\item[Acteur du MdE :] DAB
\item[CdU :] acheter une boisson

\item[Message :] \\
PanneauInformation ::{afficherMonnaie, monnaieInsuffisante, boissonNonDispo\}\\
RecepMonnaie::{relacherMonnaie}

\item[Pré :] 
\item[Post :] \\
	self.montantReçu = 0 and\\
	self.récepMonnaie ^relâcherMonnaie() and\\
	self.panneauInformation ^afficherMonnaie(0) and\\
	self.panneauInformation ^monnaieInsuffisante(false) and\\
	self.panneauInformation ^ boissonNonDispo(false) \\

\end{description}


\LQModel{
    Un MdC découle un MdE donc on a autant de MdC que de MdE. Ils utilisent les
    même acteurs et donc la convention de nommage utilisée pour le MdE doit
    être la même pour le MdC.
}

\end{LQ}
