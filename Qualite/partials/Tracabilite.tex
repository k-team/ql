\begin{LQ}{Traçabilité}

\LQDescription{
    	Pour notre étude, nous mettons en place une procédure de traçabilité. L’objectif est de pouvoir faire le lien entre les différents modèles et objets du modèle tout en gardant une certaine cohérence tout au long du projet. Ainsi chaque modèle est relié aux par certains critères que l’on décrit ci -dessous.

}

\textbf{Modèle de l'environnement}\\
Pour ce modèle, on spécifie les interactions entre le système et ces acteurs. \\
Les acteurs du MdE sont réliés au acteurs du CdU (dans le modèle textuelle du MdE). De plus chaque action correspond appartient à un CdU. Cette spécification au niveau du MdO associé.\\
\\

\textbf{Modèle des concepts}\\
Pour ce modèle, on détaille précisément , sous forme de diagramme de classe le système. Les acteurs du MdE sont réutilisés pour contruire ce modèle. Ainsi un même système relie MdE, MdP, MdC et MdO.\\
\\


\textbf{Modèle du protocole}\\
Pour ce modèle, on représente un diagramme d’état transition. Les transitions sont les actions du MdE ayant le même sous système. Ainsi un même système relie MdE, MdP, MdC et MdO.\\
\\



\textbf{Modèle des opérations}\\
Pour ce modèle, on décrit les différentes opérations évoqué dans le MdP ayant le même sous système. Ainsi un même système relie MdE, MdP, MdC et MdO.\\

A chaque opération , on spécifie dans l’item correspondant le CDu associé et l’acteur du MdE dont il provient.\\
\\

\end{LQ}
