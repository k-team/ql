\begin{EM}{Système Web}
    \EMActor{BtnAssignBalance}{Un bouton pour assigner un produit à une balance
        existante.}
    \begin{EMUCActor}{Utilisateur}
        \EMOperation{assignBal(bal, art, seuil, qte)}{Le bouton a été activé pour
            la balance $bal$ avec l'article $art$ et comme configuration le couple
            $(seuil, qte)$, respectivement le seuil et la quantité.}{}
    \end{EMUCActor}

    \EMActor{BtnConfigBalance}{Un bouton pour configurer une balance}
    \begin{EMUCActor}{Utilisateur}
        \EMOperation{configBal(bal, seuil, qte)}{Le bouton a été activé pour la
            balance $bal$ avec la configuration $(seuil, qte)$.}{}
    \end{EMUCActor}

    \EMActor{BtnScan}{Un bouton pour identifier les balances associées au client}
    \begin{EMUCActor}{Utilisateur}
        \EMOperation{scan()}{Demande d'identification de la part d'un contrôleur
            externe (liaison entre les deux toujours pas établie). Répond par une
            confirmation si la clé du système balance correspond à celui du
            contrôleur.}{}
    \end{EMUCActor}

    \EMActor{BtnVoirCommande}{Un bouton pour afficher la liste des commandes
        courante associée à l'utilisateur.}
    \begin{EMUCActor}{Utilisateur}
        \EMOperation{voirCommande()}{Le bouton à été actionné.}{}
    \end{EMUCActor}

    \EMActor{BtnValiderCommande}{Un bouton pour valider la liste des commandes
        courante associée à l'utilisateur.}
    \begin{EMUCActor}{Utilisateur}
        \EMOperation{validerCommande()}{Le bouton à été actionné.}{}
    \end{EMUCActor}

    \EMActor{InterfaceCatalogue}{Interface client pour ajouter un produit à la
        liste des commandes de l'utilisateur.}
    \begin{EMUCActor}{Utilisateur, Controleur}
        \EMOperation{listeArticle()}{La liste des produits est envoyée au client.}{}
        \EMOperation{produitNonStock(art)}{L'article $art$ n'est plus en stock.}{}
    \end{EMUCActor}

    \EMActor{SystèmeBalance}{L'interface balance qui notifie un manque en quantité
        de l'article.}
    \begin{EMUCActor}{Balance, Controleur}
        \EMOperation{alarm(art, qte)}{Une quantité $qte$ est demandée pour l'article
            $art$ associé à la balance afin d'être ajoutée à la commande.}{}
    \end{EMUCActor}

    \EMActor{ListeCommande}{La liste des commandes de l'utilisateur divisée en
        articles, avec chacun ses caractéristiques.}
    \begin{EMUCActor}{Utilisateur, Controleur}
        \EMOperation{estArticleCoché(art)}{Un article $art$ de la liste des commande
            a été coché ou décoché.}{}
        \EMOperation{changerQuantité(art, qte)}{Demande de modification de la
            quantité de l'article $art$ par la quantité $qte$.}{}
        \EMOperation{majCommande()}{Demande du $Système Web$ de mettre à jour la
            liste des commandes de l'utilisateur.}{}
    \end{EMUCActor}

    \EMActor{PanneauInformation}{Un écran pour afficher les différentes
        informations sur l'état du système (du texte, des lumières, des graphiques,
        des boutons, ...).}
    \begin{EMUCActor}{Utilisateur, Controleur}
        \EMOperation{afficherErreur(err)}{Demande du $Système Web$ d'afficher
            l'erreur $err$ rencontrée.}{}
        \EMOperation{produitNonDispo(art, b)}{Si $b$ est vrai, le $Système Web$
            demande l'affichage d'un texte annonçant que l'article $art$ n'est pas
            disponible en stock. Sinon, le $Système Web$ n'affiche rien.}{}
        \EMOperation{commandeValidée(b)}{Si $b$, le $Système Web$ demande
            l'affichage d'un texte annonçant que la commande a été validée. Sinon,
            le $Système Web$ n'affiche rien.}{}
        \EMOperation{afficherBalance(bal)}{Demande du $Système Web$ d'afficher la
            balance $bal$ sélectionnée depuis la liste des balances de
            l'utilisateur.}{}
        \EMOperation{afficherCommande(cmd)}{Demande du $Système Web$ d'afficher la
            liste des commandes $cmd$ de l'utilisateur.}{}
        \EMOperation{afficherFacture(cmd, prixTotal)}{Demande du $Système Web$
            d'afficher la facture relative à la commande $cmd$ avec le prix total
            $prixTotal$ liée à la commande.}{}
        \EMOperation{confirmAssignBal()}{L'assignement du produit a bien eu lieu.}{}
        \EMOperation{erreurAssignBal()}{Une erreur a eu lieu pendant l'assignement
            de la balance.}{}
        \EMOperation{confirmConfigBal()}{La configuration a bien eu lieu.}{}
        \EMOperation{erreurConfigBal()}{Une erreur a eu lieu durant la
            configuration.}{}
    \end{EMUCActor}
\end{EM}
