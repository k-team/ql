\begin{EM}{Réceptacle}
    \EMActor{PaveNum}{Pavé numérique type standard, composé de chiffres (valeur
        allant de 0 à 9), d'un bouton "valider", et d'un bouton "annuler".}
    \begin{EMUCActor}{}
        \EMOperation{rentrerCode(n)}{Le code $n$ est vérifié.}{}
        \EMOperation{modifierCode(n)}{Le code $n$ est fixé en tant que nouveau
            code.}{}
    \end{EMUCActor}

    \EMActor{capteurBadge}{Capteur de badge pour détecter la présence du badge de
        l'utilisateur (ou celui du technicien).}
    \begin{EMUCActor}{}
        \EMOperation{vérificationBadge(b)}{La valeur du badge est vérifiée.}{}
    \end{EMUCActor}

    \EMActor{diode}{Diode pouvant prendre une couleur verte ou rouge pendant un
        temps donné.}
    \begin{EMUCActor}{}
        \EMOperation{diodeVerte()}{Demande à la diode de devenir verte pour 3
            secondes.}{}
        \EMOperation{diodeRouge()}{Demande à la diode de devenir rouge pour 3
            secondes.}{}
    \end{EMUCActor}

    \EMActor{panneauAffichage}{Panneau d'affichage pour l'affichage d'informations
        générales sur l'état du système.}
    \begin{EMUCActor}{}
        \EMOperation{msgAccueil()}{Demande d'affichage du message d'accueil.}{}
        \EMOperation{msgCodeIncorrect(e)}{Si $e$ est vrai, demande d'affichage d'un
            message indiquant que le code rentré est incorrect. Sinon, demande du
            réceptacle d'effacer le message.}{}
        \EMOperation{msgBadgeIncorrect(e)}{Similaire à $msgCodeIncorrect(e)$.}{}
        \EMOperation{msgModifierCode(e)}{Similaire à $msgCodeIncorrect(e)$.
            L'effacement est suivi d'une demande d'affichage d'un message
            demandant la fermeture.}{}
        \EMOperation{msgHorsService(e)}{Similaire à $msgCodeIncorrect(e)$.}{}
    \end{EMUCActor}

    \EMActor{supportDrone}{}
    \begin{EMUCActor}{}
        \EMOperation{dronePresent()}{Le support indique que le drône est présent.}{}
        \EMOperation{bloquerDrone(e)}{Si $e$ est vrai, une demande est envoyée au
            support pour bloquer le drône. Sinon, une demande est envoyée au
            support pour relâcher le drône.}{}
    \end{EMUCActor}

    \EMActor{porteLaterale}{Porte latérale du réceptacle, utilisée par le
        client pour récupérer les colis. Ne peut être ouverte que manuellement.}
    \begin{EMUCActor}{}
        \EMOperation{porteFerme()}{Signale au réceptacle que la porte vient d'être
            fermée.}{}
        \EMOperation{bloquerPorte()}{Demande du réceptacle de bloquer la porte
            latérale.}{}
        \EMOperation{débloquerPorte()}{Demande du réceptacle de débloquer la porte
            latérale.}{}
    \end{EMUCActor}

    \EMActor{panneauSuperieur}{Panneau supérieur du réceptacle, utilisé par le
        drone pour déposer les colis. Peut être ouverte électroniquement ou
        manuellement.}
    \begin{EMUCActor}{}
        \EMOperation{panneauFerme()}{Signale au réceptacle que le panneau supérieur
            vient d'être fermé.}{}
        \EMOperation{bloquerPanneau()}{Demande du réceptacle de bloquer le panneau
            supérieur.}{}
        \EMOperation{debloquerPanneau()}{Demande du réceptacle de débloquer le
            panneau supérieur.}{}
        \EMOperation{ouvrirPanneau()}{Demande du réceptacle d'ouvrir du
            panneau.}{}
        \EMOperation{fermerPanneau()}{Demande du réceptacle de fermer le panneau
            supérieur.}{}
    \end{EMUCActor}

    \EMActor{capteurRFID}{Un capteur RFID permettant de vérifier la présence d'un
        colis à l'intérieur du réceptacle.}
    \begin{EMUCActor}{}
        \EMOperation{colisPresent()}{Le capteur indique s'il a détecté la présence
            d'un colis à l'intérieur du réceptacle.}{}
    \end{EMUCActor}

    \EMActor{systemeDrone}{Correspond au système du drone en liaison avec le
        réceptacle.}
    \begin{EMUCActor}{}
        \EMOperation{verificationColis(idDestinataire)}{Signal du drone demandant
            la vérification de la correspondance entre un colis et le
            réceptacle.}{}
        \EMOperation{ouverturePanneau()}{Signal du drone demandant l'ouverture du
            panneau supérieur du réceptacle.}{}
        \EMOperation{fermeturePanneau()}{Signal du drone demandant la fermeture du
            panneau supérieur du réceptacle.}{}
        \EMOperation{timeoutVerification()}{Signal du drone indiquant que celui-ci
            n'a pas reçu la confirmation de l'action demandée dans le temps
            voulu.}{}
        \EMOperation{receptaclePlein()}{Signal du réceptacle indiquant au drone que
            le réceptacle est plein.}{}
        \EMOperation{colisCorrect(e)}{Si $e$ est vrai, signale que le colis
            correspond bien à ce réceptacle. Sinon, signale que le colis ne
            correspond pas à ce réceptacle.}{}
        \EMOperation{signalReceptacle()}{Signal du réceptacle pour indiquer au
            drone la position du réceptacle pour que celui-ci se pose.}{}
        \EMOperation{sendConfirmation(msg)}{Signal du receptacle pour indiquer
            qu'une action a bien été réalisée. Cette action est indiqué par le
            message $msg$ (sous la forme d'une énumération).}{}
    \end{EMUCActor}

    \EMActor{systemeClient}
    \begin{EMUCActor}{}
        \EMOperation{signalerReception()}{Signal du réceptacle au système client
            pour indiquer qu'il y a un colis à récupérer dans le réceptacle.}{}
    \end{EMUCActor}
\end{EM}
